\documentclass{article}
\usepackage{amsmath}
\usepackage{amsfonts}
\usepackage{mathtools}
\usepackage{color}

\DeclarePairedDelimiter\abs{\lvert}{\rvert}%
\DeclarePairedDelimiter\norm{\lVert}{\rVert}%

\title{Analyzing Orientation Using Spectral Methods}
\author{Mark Kittisopikul}

\begin{document}

\maketitle

\section{Notation}
\subsection{Fourier Transforms}
For a function, $f(x)$, the Fourier transform with respect to $ x $ as a function of the frequency, $ \omega $, is $ \mathcal{F}_x\{f\}(\omega) $. This can also be written in hat notation as $ \hat{f}(\omega) $ where the original domain over $ x $ is assumed based on the function definition. For example, for a function $ I(x,y) $, the 2D Fourier transform over $ x $ and $ y $ can be written as follows $ \mathcal{F}_{xy}\{I\}(f_x,f_y) = \mathcal{F}\{I\}(f_x,f_y) = \widehat{I}(f_x,f_y) $ where $ f_x $ and $ f_y $ are the x and y component of the frequency in reciprocal space. The subscript of $ \mathcal{F} $ explicitly denotes the variables over which .

\subsection{Polar Coordinates}
A function of Cartesian coordinates, $ \Phi(x,y) $, may be defined in polar coordinates $ \Phi(r,\theta) $. In this case, take $ \Phi(r,\theta) $ to mean $\Phi(x = r cos(\theta),y = r sin(\theta) ) $ and $ \Phi(x,y) = \Phi(r = \sqrt{x^2+y^2},\theta = tan^{-1}(y/x)) $. Unless explicitly stated otherwise, $ \widehat{\Phi} $ indicates the Fourier Transform in Cartesian coordinates as opposed to polar coordinates. That is $ \widehat{\Phi} = \mathcal{F}_{xy}\{\Phi(x,y)\} $.

\subsection{Parameters}
A function with a parameter such as $ m $ will be listed with parameters after the semicolon. For example for a function representing a line in space, $ y = f(x; m,b) = mx+b $ where $ m $ is the slope and $ b $ is the y-intercept, $ m $ and $ b $ are the parameters.

\section{Filter Design}
The Orientation Space filter is chosen such that it is polar separable in the frequency domain:
\begin{eqnarray}
    \Phi(f,\theta; f_c, K, \theta_0) & = & \Phi_f(f; f_c) \Phi_\theta(\theta; K, \theta_0) \\
    & \mbox{where} & \nonumber \\
    \Phi_f(f; f_c) & = & \frac{f}{f_c} \exp(-\frac{f^2}{2f_c^2}+\frac{1}{2})  \\
    & \mbox{and} & \nonumber \\
    \Phi_\theta(\theta; K, \theta_0) & = & \exp \left(\frac{-(\theta-\theta_0)^2}{2} \frac{(2K+1)^2}{\pi^2} \right)
\end{eqnarray}
For ridges and edges the angular component is duplicated at $ \theta + \pi $ but with different signs:
\begin{eqnarray}
    \Phi_{\theta,\mbox{ridge}}(\theta) & = & \Phi_{\theta}(\theta) + \Phi_{\theta}(\theta+\pi) \\
    \Phi_{\theta,\mbox{edge}}(\theta) & = & \Phi_{\theta}(\theta) - \Phi_{\theta}(\theta+\pi)
\end{eqnarray}

Due to linearity of convolution converting to the edge or ridge form can be done before or after this filter is convolved with the signal of interest.

For this report a radial component parameter $ K_f = (\frac{f_c}{b_f})^2 $ was set to unity for simplicity. The scale parameter $ f_c $ which sets the central frequency will not be referred to further in this report.

The parameter $ K $ controls the angular order of the filter and sets the standard deviation of the Gaussian to be $ \frac{\pi}{2K+1} $. This bandlimits the angular response signal since

\begin{eqnarray}
	\widehat{\Phi_{\theta}}(\omega ; K) & = & \sqrt{2\pi}\exp\left(-\frac{\omega^2}{2} \left(\frac{2\pi}{2K+1}\right)^2\right)\\
	\lim_{K\to\infty} \widehat{\Phi_{\theta}}(K+1 ; K) & = & \sqrt{2\pi} \exp\left(-\frac{\pi^2}{2}\right) \\
	& = & 7.2 \times 10^{-3} \sqrt{2\pi} \\
    & = & 7.2 \times 10^{-3} \quad \widehat{\Phi_{\theta}}(0 ; K) \\
    \lim_{K\to\infty} \frac{\widehat{\Phi_{\theta}}(K+1 ; K)}{\widehat{\Phi_{\theta}}(0 ; K)} & = & 7.2 \times 10^{-3} \approx  0
\end{eqnarray}

This means that the (K+1)th Fourier coefficient of the filter is two to three orders of magnitude less than the 0th coefficient. It is the equivalent of approximating a Gaussian by only calculating the distribution up to $ \pi \sigma > 3 \sigma $. Because the magnitude of the coefficients are near zero past a certain $ K_{bandlimit} $, we say the the result of the convolution of this filter is bandlimited and we can then express the signal as a finite Fourier series as done below.

While van Ginkel uses $ K \in \mathbb{Z} $ and particularly $ K \geq 3 $, $ K $ can be a real number such that $ K \in (-\frac{1}{2},+\infty) $ where
\begin{eqnarray}
\Phi_{\theta}(\theta; K = -0.5,\theta_0) & \equiv & 1 \\
\Phi_{\theta}(\theta; K = +\infty,\theta_0) & = & \delta(\theta_0)
\end{eqnarray}
and then $K_{bandlimit} = \max\{\lceil K \rceil,1\} $


\subsection{Alternatives for Angular Component}

It would be useful if rather than being approximately zero that $ \widehat{\Phi}_{\theta}(\omega,K) $ was exactly zero for $ \omega > K_{bandlimit} $. Such a angular component can derived from the family of directional derivatives as follows:

\begin{eqnarray}
    \Phi_{\theta,\mbox{wavelet}}(\theta; K) & = & \cos^K(\theta) \\
    \widehat{\Phi_{\theta,\mbox{wavelet}}}(\omega; K) & = & 
    \begin{cases}
        0                               & \abs{\omega} > K_{bandlimit} \\
        0                               & K \mbox{ odd and }  \omega \mbox{ even} \\
        K \choose (\omega+K)/2 & K \mbox{ odd and }  \omega \mbox{ odd} \\
        0                               & K \mbox{ even and } \omega \mbox{ odd} \\
        K \choose (\omega+K)/2 & K \mbox{ even and } \omega \mbox{ even}
    \end{cases}
\end{eqnarray}

This form of an angular component arises naturally out of directional derivatives as well as the Gabor filters.

where $ {K \choose n} = \frac{K!}{(K-n)!n!} $. Here, however, $ K \in \mathbb{Z} $. Also even $ K $ correspond to ridge filters and odd $ K $ correspond to edge filters, meaning that the bandwidth cannot be matched for edges and ridges. The Fourier transform of this angular component follows a binomial distribution with either the odd or even coefficients being zero depending on if K is even or odd, respectively.

More generally, $ \Phi_{\theta,\mbox{general}} $ could be any function expressible as a finite Fourier series. The goal, however, in this case is to obtain a function that as orientation-selective as possible within a certain bandlimit, thus providing the best orientation resolution. The Gaussian, while only approximately bandlimited, uniquely optimizes the orientation-bandwidth and frequency-bandwidth simultaneously.



\section{Filter Response}

The filter response is obtained by doing a 2D convolution with the image in Cartesian coordinates. By the convolution theorem, this 2D convolution can be conducted by an element-wise product of the 2D xy Fourier transforms of the image and filter.

\begin{eqnarray}
    R_{\theta_0,K}(x,y) & = & I(x,y) \ast \hat{\Phi}(x,y; K) \\
    \widehat{R_{\theta_0,K}}(f_x,f_y) & = & \hat{I}(f_x,f_y) \Phi(f_x,f_y; K, \theta_0) \\
    \widehat{R_{\theta_0,K}}(f,\theta) & = & \hat{I}(f,\theta) \Phi(f,\theta; K, \theta_0)
\end{eqnarray}

The rest of this report will focus on the response of a single point, $ (x_0,y_0) $, across $ \theta_0 $.

\begin{eqnarray}
    p(\theta_0,K) & = & R_{\theta_0,K}(x_0,y_0)
\end{eqnarray}

Note that because of this construction we can think of $ p(\theta,K) $ as a convolution of the an unbandlimited ($K = \infty $) angular response signal and the angular component of the filter.

\begin{eqnarray}
    p(\theta,K) & = & p(\theta,+\infty) \ast_\theta \Phi_\theta(\theta,K) \\
    \mbox{where } \Phi_\theta(\theta,+\infty) & = & \delta({\theta}) 
\end{eqnarray}
By this formulation, we then calculate the angular response for any $ K_2 $ from an angular response $ K_1 $ provided that $ K_2 < K_1 $:
\begin{eqnarray}
    \mathcal{F}_\theta\{ p \}(\omega,K) & = & \mathcal{F}_\theta(\omega,+\infty) \circ_\omega \widehat{\Phi_\theta}(\omega,K) \\
    \mathcal{F}_\theta\{ p \}(\omega,K_2) & = & \mathcal{F}_\theta(\omega,+\infty) \circ_\omega \widehat{\Phi_\theta}(\omega,K_1) \circ_\omega \frac{ \widehat{\Phi_\theta}(\omega,K_2) }{ \widehat{\Phi_\theta}(\omega,K_1)  } \\
    & = &  \mathcal{F}_\theta\{ p \}(\omega,K_1) \circ_\omega \frac{ \widehat{\Phi_\theta}(\omega,K_2) }{ \widehat{\Phi_\theta}(\omega,K_1)  } \\
    & = & \mathcal{F}_\theta\{ p \}(\omega,K_1) \nonumber\\ 
    & \circ_\omega & \exp\left(- \frac{(2\pi\omega)^2}{2}  \left[ \left(\frac{1}{2K_2+1}\right)^2 - \left(\frac{1}{2K_1+1}\right)^2 \right] \right) \label{eqn:phi_k2_k1} \\
    & = & \mathcal{F}_\theta\{ p \}(\omega,K_1)
    \circ_\omega   \exp\left(- \frac{(2\pi\omega)^2}{2}  \left[ \frac{N_1^2 - N_2^2}{N_1^2 N_2^2} \right] \right) \\
    N_1 & = & 2K_1 + 1 \\
    N_2 & = & 2K_2 + 1 
\end{eqnarray}



\section{Fourier Series Expansion of Response}
Because of the bandlimited design of the angular component of the Orientation Space filter, the $\pi$-periodic ridge orientation response can be well described by a finite Fourier Series, which can also be expressed in polynomial form.
\begin{eqnarray}
    p(\theta,K) & = & \sum_{n=-K}^{n=K} C_{n,K} \exp(2 i n \theta) \\
    & = & \sum_{n=-K}^{n=K} C_{n,K} \left[z(\theta)\right]^n \quad \mbox{where } z(\theta) = \exp(2 i \theta)
\end{eqnarray}
The coefficients $ C_{n,K} $ can be obtained by applying a Fourier Transform to a $ 2K+1 $ sampling of the response signal.
\begin{eqnarray}
    C_{n,K} & = & \sum_{q=-K}^{q=K} p(\theta_q,K) \exp(-2 i n \theta_q) \quad \mbox{where } \theta_q = \frac{q \pi}{2K+1} \\ 
    & = & \mathcal{F}_\theta \{ p \}( n, K)
\end{eqnarray}

From the prior section, it should be clear we can obtain $ C_{n,K_2} $ from $ C_{n,K_1} $ when $ K_2 < K_1 $. 

\section{Finding Orientation Local Maxima in Response}

We would like to find the local maxima as a function orientation because these orientations correspond to the direction of lines in the image. From basic calculus, the orientation local maxima in the response occurs when
\begin{eqnarray}
	\frac{\partial p(\theta,K)}{\partial \theta} & = & 0 \\
	\mbox{and } \frac{\partial^2 p(\theta,K)}{\partial \theta^2} & < & 0
\end{eqnarray}

Since we can express $ p(\theta,K) $ as a Fourier series, we can write an expression for it's derivatives:

\begin{eqnarray}
	\frac{\partial p(\theta,K)}{\partial \theta} & = & \sum_{n=-K}^{n=K} 2 n i C_{n,K} z^n \\
	\frac{\partial^2 p(\theta,K)}{\partial \theta^2} & = & \sum_{n=-K}^{n=K} - 4 n^2 C_{n,K} z^n \\
	\frac{\partial^3 p(\theta,K)}{\partial \theta^3} & = & \sum_{n=-K}^{n=K} - 8 n^3 i C_{n,K} z^n \\
	\frac{\partial^d p(\theta,K)}{\partial \theta^d} & = & \sum_{n=-K}^{n=K} (2ni)^d C_{n,K} z^n \\
	\mbox{where } z & = & \exp(2i\theta)
\end{eqnarray}

{\color{red} Here I used $ z = \exp(2i\theta) $ which causes the 2 to propagate through the derivatives. It might be easier to use another variable that is $ 2 \pi $ periodic and then transofrm back to $ \pi $ periodicity later, which is fhow I actually do the computation.}

For polynomials of degree less than five, explicit formula exist. For polynomials of degree five or higher, the Abel-Ruffini theorem states there is no general algebraic solution for an arbitrary set of algebraic coefficients.
To find the zeros of the first derivative, a general Frobenius companion matrix approach can be used. The idea is to form a companion matrix as follows
\[
	M = \begin{bmatrix}
		\frac{K-1}{K} C_{K-1,K}/C_{K,K}  & \frac{K-2}{K} C_{K-2,K}/C_{K,K} & \frac{K-3}{K} C_{K-3,K}/C_{K,K} & \dots & -C_{-K,K}/C_{K,K} \\
		1                  & 0                 & 0                 & \dots & 0 \\
		0                  & 1                 & \ddots            & \ddots & 0 \\
		0                  & 0                 & \ddots            & \ddots & 0 \\
		\vdots             & \ddots            & \ddots            & \ddots& \vdots \\
		0                  & \dots                 & 0                 & 1     & 0 
	\end{bmatrix}
\]

Note that the characteristic polynomial of M is equal to the first derivative of the response:
\[ \det(\lambda I - M ) = \frac{\partial p}{\partial \theta} z^{K} \]
Thus the eigenvalues of M are the roots of the first derivative:
\begin{eqnarray}
	M \lambda & = & M v \\
	\frac{\partial p(\lambda,K)}{\partial \theta} & = & =0
\end{eqnarray}
We are only interested in $ \lambda = \exp(2i\theta) $ such that $ \theta $ is real. This occurs when $ \lambda $ lies on unit circle in the complex plane. This can be determined by ensuring that $ \abs{ \log \norm{\lambda} } $ is sufficiently close to zero.



\section{Diffusion of Orientation Space}

In order to understand the effect of the parameter $ K $ it is useful to make an analogy to physical transport phenonmenon of diffusion..

Equation \ref{eqn:phi_k2_k1} shows that if we have the Fourier series for $ p(\theta,K_1) $ then we obtain the Fourier series $ p(\theta,K_2) $ through convolution with a Gaussian. This is reminscent of the heat equation and linear scale space. We can then make an analogy between time, $ t $, in diffusion and the angular order, $ K $.
\begin{eqnarray} 
	\mbox{Let } t & = & \frac{1}{(2K+1)^2} \\
	\mbox{then } \frac{dt}{dK} & = & \frac{-4}{(2K+1)^3} 
\end{eqnarray}
We can rewrite equation \ref{eqn:phi_k2_k1} as follows:
\begin{eqnarray}
	\mathcal{F}_\theta \{ p \} (\omega,t_2) & = & \mathcal{F}_\theta \{ p \} (\omega,t_1)  \circ_\omega \exp\left(- \frac{(2\pi\omega)^2}{2}  \left[ t_2 - t_1 \right] \right) \\
	p(\theta,t_2) & = & \sum_{n=-K}^{n=K} C_{n,t_1} \exp(-2\pi^2 n^2 \left[t_2-t_1\right]) \left[ z(\theta) \right]^n  \\
	\frac{\partial p(\theta,t_2)}{\partial t_2} & = & -2 \pi^2 \sum_{n=-K}^{n=K} n^2 C_{n,t_1} \exp(-2\pi^2 n^2 \left[t_2-t_1\right]) \left[ z(\theta) \right]^n  \\
	\frac{\partial p(\theta,t_2)}{\partial t_2} & = & \frac{\pi^2}{2} \frac{\partial^2 p(\theta, t_2)}{\partial \theta^2} \label{eqn:heateqn} \\
	\frac{\partial p(\theta,t_2)}{\partial K_2} & = & \frac{\partial p(\theta,t_2)}{\partial t_2} \frac{dt_2}{dK_2}  \\
	& = &  \frac{\partial p(\theta,t_2)}{\partial t_2} \frac{-4}{(2K_2+1)^2} \\
	& = &  \frac{-2 \pi^2 }{(2K_2+1)^2} \frac{\partial^2 p(\theta, t_2)}{\partial \theta^2} 
\end{eqnarray}

Essentially, the angular order is analogous to time moving backwards. In this analogy then, the orientation response responds to decreasing angular order in a fashion similar to diffusion with periodic boundary conditions.

\subsection{Causality}
One important property that is inherited from this analogy with diffusion is causality of local extrema. From diffusion, it is known that local extrema are not enhanced over time and that new local extrema do not arise as time progresses. Here this means as the angular order is reduced no new local maxima in orientation will occur.

\subsection{Tracing Orientation Local Maxima}
Due to causality, this means that if we know the location of local maxima at $ t = t_1 $ or $ K = K_1 $ then we may be able to obtain the location of the local maxima at $ t = t_2 $ or $ K = K_2 $ without having to solve another eigen problem. A key insight here is to recognize that all partial derivatives of $ p(\theta,t) $ obey the heat equation (Equation \ref{eqn:heateqn}).

Suppose we have a local maximum at $ \theta_m $. Then
\begin{eqnarray}
	\frac{\partial p(\theta_m(t),t)}{\partial \theta} & = & 0 \\
	\frac{\partial^2 p(\theta_m(t),t)}{\partial \theta^2} & < & 0 
\end{eqnarray}
From the heat equation, we know that evolution in time of the first derivative with respect to $ \theta $ will be driven by the third derivative with respect to $ \theta $. Similarly with the second derivative evolves in time according to the fourth derivative with respect to theta.
\begin{eqnarray}
	\frac{\partial^2 p(\theta_m(t),t)}{\partial t \, \partial \theta} & = & \frac{\pi^2}{2} \frac{\partial^3 p(\theta_m(t),t)}{\partial \theta^3} \\
	\frac{\partial p(\theta_m(t),t+dt)}{\partial \theta} - \frac{\partial p(\theta_m(t),t)}{\partial \theta} & = & \frac{\pi^2}{2} \frac{\partial^3 p(\theta_m(t),t)}{\partial \theta^3} dt \\
	\frac{\partial p(\theta_m(t),t+dt)}{\partial \theta} & = & \frac{\pi^2}{2} \frac{\partial^3 p(\theta_m(t),t)}{\partial \theta^3} dt \\
	\frac{\partial^3 p(\theta_m(t),t)}{\partial t \, \partial \theta^2} & = & \frac{\pi^2}{2} \frac{\partial^4 p(\theta_m(t),t)}{\partial \theta^4}  \\
	\frac{\partial^2 p(\theta_m(t),t+dt)}{\partial \theta^2} - \frac{\partial^2 p(\theta_m(t),t)}{\partial \theta^2} & = & \frac{\pi^2}{2} \frac{\partial^4 p(\theta_m(t),t)}{\partial \theta^4} dt \\
	\frac{\partial^2 p(\theta_m(t),t+dt)}{\partial \theta^2} =  \frac{\partial^2 p(\theta_m(t),t)}{\partial \theta^2} & + & \frac{\pi^2}{2} \frac{\partial^4 p(\theta_m(t),t)}{\partial \theta^4} dt
\end{eqnarray}
Having the first and second derivatives at $ \theta_m(t) $ with repect $ \theta $ allows us to make a first order approximation for $ \theta_m(t+dt) $ by drawing a line through $ ( \theta_m(t) , \frac{\pi^2}{2} \frac{\partial^3 p(\theta_m(t),t)}{\partial \theta^3} dt ) $ with slope $ \frac{\partial^2 p(\theta_m(t),t+dt)}{\partial \theta^2} $. Locating $ \theta_mt(t+dt) $ is then just a matter of finding the new $\theta$-intercept.
\begin{eqnarray}
	\frac{\partial p(\theta_m(t+dt),t+dt)}{\partial \theta} & = & 0 \\
	\frac{\partial^2 p(\theta_m(t+dt),t+dt)}{\partial \theta^2} & < & 0  \\
	\frac{\partial^2 p(\theta_m(t),t+dt)}{\partial \theta^2} \left[ \theta_m(t+dt) - \theta_m(dt) \right] & = & - \frac{\partial p(\theta_m(t),t+dt)}{\partial \theta} \\
	\frac{ \theta_m(t+dt) - \theta_m(dt) }{dt} & = & - \frac{ \frac{\partial p(\theta_m(t),t+dt)}{\partial \theta}  }{ \frac{\partial^2 p(\theta_m(t),t+dt)}{\partial \theta^2}  } \frac{1}{dt} \\
	\lim_{dt \to 0 } \frac{ \theta_m(t+dt) - \theta_m(dt) }{dt} & = & \nonumber \\
	\lim_{dt \to 0 } - \frac{ \displaystyle \frac{\pi^2}{2} \frac{\partial^3 p(\theta_m(t),t)}{\partial \theta^3}  }{ \displaystyle \frac{\partial^2 p(\theta_m(t),t)}{\partial \theta^2} \nonumber
	                                                                       + \frac{\pi^2}{2} \frac{\partial^4 p(\theta_m(t),t)}{\partial \theta^4} dt  } \\
\end{eqnarray}
\begin{eqnarray}
									       \frac{d \theta_m(t)}{dt} & = & 
									       - \frac{ \displaystyle \frac{\pi^2}{2} \frac{\partial^3 p(\theta_m(t),t)}{\partial \theta^3}  }{ \displaystyle \frac{\partial^2 p(\theta_m(t),t)}{\partial \theta^2} }  \\
									       \frac{d \theta_m(K)}{dK} & = & 
									       \frac{2 \pi^2}{(2K+1)^3}
									       \frac{ \displaystyle \frac{\partial^3 p(\theta_m(K),K)}{\partial \theta^3}  }{ \displaystyle \frac{\partial^2 p(\theta_m(K),K)}{\partial \theta^2} } 
\end{eqnarray}

This suggests that the position of the local maxima of space encounters a critical point with respect to $ K $ when the third derivative is zero. Additionally, should the second derivative become zero or positive, then the local maximum ceases by definition to exist for smaller K.

It thus may be interesting to note the K when $ \frac{\partial p(\theta_m,K)}{\partial \theta} = 0 $ and either $ \frac{\partial^2 p(\theta_m,K)}{\partial \theta^2} = 0 $ or $ \frac{\partial^3 p(\theta_m,K)}{\partial \theta^3} = 0 $.


\end{document}
