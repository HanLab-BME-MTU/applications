\documentclass{article}
\usepackage{amsmath}
\usepackage{amsfonts}
\usepackage{mathtools}
\usepackage{color}

\DeclarePairedDelimiter\abs{\lvert}{\rvert}%
\DeclarePairedDelimiter\norm{\lVert}{\rVert}%

\title{Generalizing the Results to a Framework based upon the Riesz Transform}
\author{Mark Kittisopikul}

\begin{document}

\maketitle

\section{Introduction}
Previously we have defined an orientation selective filter, $ \Phi_\theta(\omega; K) $ that applies to the spatial frequency domain after a 2D Fourier transform over the x and y coordinates. We then performed a 1D Fourier Transform over $ \theta $ to obtain $ \widehat{\Phi_\theta(\omega; K)} $. We saw that $ \Phi_\theta $ has Gaussian form in both the original spatial-frequency domain over $ (f,\theta) $ as well as in the orientation-frequency $ \omega $ domain.

We also discussed an alternative angular component $ \Phi_{\theta,\mbox{wavelet}}(\theta; K) $ whose 1D Fourier transform over has the form of a binomial distribution in the orientation-frequency, $ \omega $, domain. The binomial distribution is not very different than the Gaussian distribution suggesting that a common basis for these two approaches may exist in the orientation-frequency domain.

\section{Riesz transform: circular harmonics as a common basis}

Unser and Chenouard discussed a ``A Unifying Parametric Framework for 2D Steerable Wavelet Transforms'' which generalizes these two approaches. Namely they introduce a complex 2D Riesz transform that formalizes the serial 2D Fourier transform in space followed by the 1D Fourier transform in orientation.

In order to understand this framework and how the current work fits into it, we must first build up to the Riesz transform, which is an N-dimensional transform. The 1D version goes commonly by a different name called the Hilbert transform. This Hilbert transform, $ \mathcal{H} $ of a 1D function $ g(x) $ is defined as follows:
\begin{eqnarray}
    \mathcal{F}\{ \mathcal{H}\{g\}(x) \}(\nu) & = & (-i \frac{\nu}{\left| \nu \right|}) \mathcal{F}\{g(x)\}(\nu)
\end{eqnarray}
That is the Hilbert transform is just a multiplier in the frequency domain. The Riesz transform, $ \mathcal{R} $  generalizes this to N-dimensions for a function $ G(\vec{x}) \equiv G(x_1,x_2, \hdots, x_N) $
\begin{eqnarray}
    \mathcal{F}\{ \mathcal{R}\{G\}(\vec{x}) \}(\nu) & = & (-i \frac{\vec{\nu}}{\left\| \nu \right\|}) \mathcal{F}\{G(\vec{x})\}(\vec{\nu})
\end{eqnarray}
The complex 2D Riesz transform just encodes the vector, $ \nu $,  as a complex number that is naturally written in polar coordinates:
\begin{eqnarray}
    \mathcal{F}\{ \mathcal{R}\{G\}(\vec{x}) \}(\nu) & = & (\frac{\nu_x +i \nu_y}{\left\| \nu \right\|}) \mathcal{F}\{G(\vec{x})\}(\nu) \\
    \mathcal{F}\{ \mathcal{R}\{G\}(\vec{x}) \}(\|\nu\|,\angle \nu) & = & \exp(i \angle \nu) \mathcal{F}\{G(\vec{x})\}(\|\nu\|,\angle \nu) \\
    \mathcal{F}\{ \mathcal{R}\{G\}(\vec{x}) \}(\|\nu\|,\theta) & = & \exp(i \theta) \mathcal{F}\{G(\vec{x})\}(\|\nu\|,\theta) \\
    \mathcal{F}\{ \mathcal{R}^n\{G\}(\vec{x}) \}(\|\nu\|,\theta) & = & \exp(i n\theta) \mathcal{F}\{G(\vec{x})\}(\|\nu\|,\theta) \\
\end{eqnarray}
The complex Riesz transform is a multiplier in the Fourier domain that only depends on the orientation angle. The last equation above also shows that n-iterates of complex Riesz transform just modifies the multiplier.





\end{document}
